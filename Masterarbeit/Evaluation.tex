\chapter{Evaluation}
\label{sec:eval}

\section{Evaluation Setup}
\label{sec:evalsetup}









Hardware 


Für die Bearbeitung der Masterarbeit und alle damit verbundenen Berechnungen wird ein Mini Turm Computer von FUJITSU, Model CELSIUS W530 verwendet. Verbaut ist ein Intel(R) Core(TM) i7-4790 CPU @ 3.60GHz Prozessor mit 16 GB Arbeitsspeicher, die Grafikkarte GM107GL Quadro K2200 von Nvidia und eine Western Digital WD5000AAKX Blue 500GB interne Festplatte. 

Turtlebot 2
Der Turtlebot 2 ist ein mobiler Roboter, welcher für die Lehre und Forschung entwickelt wurde. Die Basis des Turtlebots ist die Plattform Kobuki. Sie besteht aus Elektromotoren, die die beiden Antriebsräder antreiben, entsprechende Sensoren für die Steuerung der Räder ( Gyroskop für die Lagebestimmung, Stoßstangen/Taster für die Kollisionsdetektion und weitere Sensoren für die Navigation )  und bietet die Möglichkeit, um externe Sensoren anzuschließen. Ebenso stellt die Plattform Netzanschlüsse ( 5V/1A, 12V/1.5A und 12V/5A ) über USB oder RX/TX für weitere Aktuatoren bereit. Als Energiequelle dient ein Lithium-Ion Akku mit 14,8V, in zwei Größen mit 2200mAh und 4400mAh erhältlich. Die maximale Geschwindigkeit wird mit 70 cm/s und die maximale Rotationsgeschwindigkeit mit 180°/s vom Hersteller angegeben. Zur Kobuki Plattform gehört auch eine Dockingstation, die zum kontaktlosen laden des Turtlebots dient.
Auf die Kobuki Plattform können flexibel je nach Anwendung mehrere Turtlebot Modulplatten aufgebaut werden, welche Platz für die Sensoren und einen Laptop zur Steuerung bieten. Als Standardsensor für die visuelle Detektion wird die Kamera Asus Xtion Pro Live verwendet. Diese Kamera beinhaltet zwei Audiosensoren, zwei RGB Sensoren ( Auflösung 1280x1024 Pixel, 30 fps ) für 3D Aufnahmen und ein Tiefenbildsensor ( Auflösung 640x480 Pixel bei 30 fps, 320x240 Pixel bei 60 fps ). Empfohlen wird ein Arbeitsabstand von 0,8m bis 3,5m.
Im Standardaufbau hat der Turtelbot etwa die Abmaße 352mm Durchmesser und eine Höhe von etwa 550mm, wobei die Höhe je nach Aufbau variiert.
Für die Aufgabe der Objekterkennung wurde für die Kamera eine neue Vorrichtung auf dem Turtlebot installiert, sodass diese ca. 120cm über dem Boden in einem Winkel von -15° in den Raum blickt. Hierdurch wird ein menschenähnlicheres Sichtbild ermöglicht und dadurch ein besserer Überblick über den Raum gewährleistet. 

